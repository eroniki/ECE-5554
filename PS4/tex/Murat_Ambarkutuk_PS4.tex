\documentclass{article}

\usepackage{listings}
\usepackage{graphicx}
\usepackage{color} %red, green, blue, yellow, cyan, magenta, black, white
\usepackage{amsfonts}
\usepackage{mathtools}
\usepackage{amsmath}
\usepackage{amssymb}
\usepackage{adjustbox}
\usepackage{hyperref}
\usepackage{float}
\usepackage{caption}
\usepackage{subcaption}
\usepackage[letterpaper, portrait, margin=1in]{geometry}

\newcommand{\listFigure}[3]{ \begin{figure}[H]
\includegraphics[width=\linewidth]{../submission/#1}
		\caption{#2\label{fig:#3}}
	\end{figure}		
}


\newcommand{\listsubFigure}[4]{
\begin{figure}
	\centering
	\begin{subfigure}[b]{width=0.4\textwidth}
		\includegraphics[width=\textwidth]{../submission/#1}
	\end{subfigure}		
	\begin{subfigure}[b]{width=0.4\textwidth}
		\includegraphics[width=\textwidth]{../submission/#2}
	\end{subfigure}		
	\caption{#3\label{fig:#4}}
\end{figure}
}
 
\definecolor{mygreen}{RGB}{28,172,0} % color values Red, Green, Blue
\definecolor{mylilas}{RGB}{170,55,241}
% % Preamble done! % Begin document
\begin{document}
\label{Cover}
	\begin{center}
	\large{ECE-5554 Computer Vision: Problem Set 3} 
	\vfill
	Murat Ambarkutuk \\ murata@vt.edu
	\vfill
	Mechanical Engineering Department,\\ Virginia Polytechnic Institute and State University
	\vfill
	\today
	\end{center}
\pagebreak 
\large{Answer Sheet}
\label{Short Answer}
\section{Short Answer}
\begin{enumerate}
	\item Scale-space is used for determining the scale of the of the
	interest points. \\ 
	In the case of using any local maxima in scale-space may not define the
	interest point for each run. The scale of the interest point may end up having
	different values. Therefore, we will have less distinctive but
	more repeatable features. \\
	On the other hand, the performance and the output of thresholding pipeline, by
	its nature, depends on the threshold value. If the threshold value is too high,
	local maxima may not pass and the result will be an empty set for the interest
	point. For that reason, using thresholding will be more distinctive (since all
	the candidates on different frames will be similar.) but less repetitive.
	\item An inlier for RANSAC algorithm when solving for epipolar lines is a
	line minimizes the epipolar constraint. ($x^T\mathbf{F}x = 0$) However, with
	uncalibrated view, not all the epipolar line will meet at the same point.
	Therefore, for each point, the distance between the line and the
	corresponding point will determine the outliers.
	\item While dense stereo matching, there are some points that should
	be considered to prevent the algorithm from failing. The first point is to
	check texture in the both images. If the scanlines end up being in the
	textureless areas, it is difficult to corresponde one image to another.
	Also, the second points that should be considered is the material types of the
	objects in the scene. If the objects are non-lambertian (appereance-wise light
	source dependency), it may become difficult to correlate two windows.
	 
	\item SIFT descriptors are 128 dimensional vectors, or a point in a
	128-dimensional feature space. Each dimension represents one of the 8
	gradient direction in 4 by 4 neighbouring subregion.
	\item If Generalized Hough Transform would be utilized with SIFT descriptors,
	the dimensions of the the Hough space would be 4. For each descriptor, the
	position of the descriptor \textit{(x and y)}, scale and orientation of the
	model will be searched should be considered to find the best model.
\end{enumerate}

\label{Programming Problem (Bag of Words)}
\section{Programming Problem (Bag of Words)}
\begin{enumerate}
	\item \textbf{Raw Descriptor Matching}
	\listFigure{rawDecriptorMatching_Chosen.png}{The Area Chosen, and the SIFT
	descriptors within in}{rawDecriptorMatching_Chosen}
	\listFigure{rawDecriptorMatching_Found.png}{The matched descriptors on the
	second image}{rawDecriptorMatching_Found}
	\item \textbf{Visualizing the Vocabulary}
	Please see the attach vocabulary folder to see all the corpus. 
	\textbf{Word 84}
		\listFigure{vocabulary1.png}{20 Patches for Word 84}{fullFrameInput-1}
		\listFigure{vocabulary2.png}{20 Patches for Word 604}{fullFrameInput-1}
		\listFigure{vocabulary3.png}{20 Patches for Word 745}{fullFrameInput-1}
	\item \textbf{Full Frame Queries}
		\\ \textbf{Set -1:}
		\listFigure{fullFrame1-Sample-1.png}{Input Image-1}{fullFrameInput-1}
		\listFigure{fullFrame1-Sample-2.png}{Retrieved Image-1}{fullFrame1Output-1}
		\listFigure{fullFrame1-Sample-3.png}{Retrieved Image-2}{fullFrame1Output-2}
		\listFigure{fullFrame1-Sample-4.png}{Retrieved Image-3}{fullFrame1Output-3}		
		\listFigure{fullFrame1-Sample-5.png}{Retrieved Image-4}{fullFrame1Output-4}
		\listFigure{fullFrame1-Sample-6.png}{Retrieved Image-5}{fullFrame1Output-5}		 
		\newpage
		\textbf{Set -2:}
		\listFigure{fullFrame2-Sample-1.png}{Input Image-2}{fullFrameInput-2}
		\listFigure{fullFrame2-Sample-2.png}{Retrieved Image-1}{fullFrame2Output-1}
		\listFigure{fullFrame2-Sample-3.png}{Retrieved Image-2}{fullFrame2Output-2}
		\listFigure{fullFrame2-Sample-4.png}{Retrieved Image-3}{fullFrame2Output-3}	
		\listFigure{fullFrame2-Sample-5.png}{Retrieved Image-4}{fullFrame2Output-4}
		\listFigure{fullFrame2-Sample-6.png}{Retrieved Image-5}{fullFrame2Output-5}	
		\newpage
		\textbf{Set -3:} 
		\listFigure{fullFrame3-Sample-1.png}{Input Image-3}{fullFrameInput-3}
		\listFigure{fullFrame3-Sample-2.png}{Retrieved Image-1}{fullFrame3Output-1}
		\listFigure{fullFrame3-Sample-3.png}{Retrieved Image-2}{fullFrame3Output-2}
		\listFigure{fullFrame3-Sample-4.png}{Retrieved Image-3}{fullFrame3Output-3}	
		\listFigure{fullFrame3-Sample-5.png}{Retrieved Image-4}{fullFrame3Output-4}
		\listFigure{fullFrame3-Sample-6.png}{Retrieved Image-5}{fullFrame3Output-5}
		%\textbf{Discussion:}\\ 
		
		\newpage
	\item \textbf{Region Queries}
	In the query images, the area containing the turkey and the fez was selected.
	Along with the query images, the output images can be seen below.
	\listFigure{regionQueries-FOI-5.png}{Query Image-1}{RegionQueryInput-1}
	\listFigure{regionQueries-FOI-13.png}{Query Image-2}{RegionQueryInput-2}
	\listFigure{regionQueries-FOI-14.png}{Query Image-3}{RegionQueryInput-3}
	\listFigure{regionQueries-FOI-15.png}{Query Image-4}{RegionQueryInput-4}
	\listFigure{regionQueries-FOI-16.png}{Output Image-1}{RegionQueryInput-5}
	\listFigure{regionQueries-Sample-1.png}{Output Image-2}{RegionQueryOutput-5}
	\listFigure{regionQueries-Sample-2.png}{Output Image-3}{RegionQueryOutput-5}
	\listFigure{regionQueries-Sample-3.png}{Output Image-4}{RegionQueryOutput-5}
	\listFigure{regionQueries-Sample-4.png}{Output Image-5}{RegionQueryOutput-5} 
	%\textbf{Discussion:} 
	Set-1 
\end{enumerate}

\end{document}
